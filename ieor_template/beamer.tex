\documentclass{beamer}
\usepackage{dirtytalk}
\mode<presentation> {
\usepackage{array}
\usepackage{url}
% The Beamer class comes with a number of default slide themes
% which change the colors and layouts of slides. Below this is a list
% of all the themes, uncomment each in turn to see what they look like.


\usetheme{Madrid}


% As well as themes, the Beamer class has a number of color themes
% for any slide theme. Uncomment each of these in turn to see how it
% changes the colors of your current slide theme.

%\usecolortheme{albatross}
%\usecolortheme{beaver}
%\usecolortheme{beetle}
%\usecolortheme{crane}
%\usecolortheme{dolphin}
%\usecolortheme{dove}
%\usecolortheme{fly}
%\usecolortheme{lily}
%\usecolortheme{orchid}
%\usecolortheme{rose}
%\usecolortheme{seagull}
%\usecolortheme{seahorse}
%\usecolortheme{whale}
%\usecolortheme{wolverine}

%\setbeamertemplate{footline} % To remove the footer line in all slides uncomment this line
%\setbeamertemplate{footline}[page number] % To replace the footer line in all slides with a simple slide count uncomment this line

%\setbeamertemplate{navigation symbols}{} % To remove the navigation symbols from the bottom of all slides uncomment this line
}
\usepackage[]{caption}
\usepackage{graphicx} % Allows including images
\usepackage{booktabs} % Allows the use of \toprule, \midrule and \bottomrule in tables
\usepackage{hyperref}
\setbeamertemplate{bibliography item}[text]
%----------------------------------------------------------------------------------------
%	TITLE PAGE
%----------------------------------------------------------------------------------------

\title[Credit Seminar]{Internet of Things in Supply Chain Management\\IE 694: Credit Seminar} % The short title appears at the bottom of every slide, the full title is only on the title page

\author[Inderjeet]{Inderjeet Singh} % Your name
\institute[IIT Bombay] % Your institution as it will appear on the bottom of every slide, may be shorthand to save space
{
Indian Institute Of Technology, Bombay \\ % Your institution for the title page
\medskip
%\textit{IE-694: Credit Seminar} % Your email address
}
\date{May 9, 2018} % Date, can be changed to a custom date

\begin{document}

\begin{frame}
\titlepage % Print the title page as the first slide
\end{frame}


%----------------------------------------------------------------------------------------
%	PRESENTATION SLIDES
%----------------------------------------------------------------------------------------

%------------------------------------------------


\begin{frame}
\frametitle{Introduction}
\begin{block}{What is Supply Chain Management}
\say{{\bfseries
Supply Chain Management(SCM) is collection of all the activities, which involves in planning, controlling and executing a product's flow, in the processes of acquiring raw material, production, transportation and distribution to final customers, in the most streamlined and cost effective way possible}}.\cite{scm}
\end{block}
So basically SCM insures the smooth flow of services and products keeping losses and costs (in terms of time, capital, labour etc.) to minimum, providing maximum possible profit to an organization by achieving maximum customer satisfaction.


\end{frame}

%------------------------------------------------

\begin{frame}
\frametitle{Internet of Things}
\begin{block}{What is Internet of Things}
\say{\bfseries
A network of devices such as computing devices, mechanical machines, digital devices, objects and even animals and humans provided with a unique identification, all these devices are then connected through the internet which allows the transfer and processing of the data generated by  \say{things}.}
\end{block}

\end{frame}

%------------------------------------------------
\begin{frame}
\frametitle{What is Internet of Things}
\begin{figure}
\centering
\includegraphics[width=12cm,height=8cm]{IOT}
\end{figure}

\end{frame}

%------------------------------------------------
\begin{frame}
\frametitle{IOT vs Industry 4.0}
\begin{figure}
\centering
\includegraphics[width=12cm,height=8cm]{iotvs}
\end{figure}
\end{frame}

%------------------------------------------------
\begin{frame}
\frametitle{Architecture of IoT}
\begin{block}{There are mainly two type of architecture exists-}
\begin{itemize}
	\item Three Layer Architecture.
	\item Five Layer Architecture.
\end{itemize}
\end{block}
\end{frame}

%------------------------------------------------
\begin{frame}
\frametitle{Three Layer Architecture}
\begin{figure}
\centering
\includegraphics[width=12cm,height=6cm]{Picture1}
\caption{Three Layer Architecture}
\end{figure}
\end{frame}
%------------------------------------------------
\begin{frame}
\frametitle{Five Layer Architecture}
\begin{figure}
\centering
\includegraphics[width=12cm,height=6cm]{Picture2}
\caption{Five Layer Architecture}
\end{figure}
\end{frame}
%------------------------------------------------

\begin{frame}
\frametitle{Major Components of IoT}

Major Components of IoT are- 
\begin{itemize}
	\item ID.
	\item Security Control.
	\item Relationship Management.
	\item Service Discovery.
	\item Meta-information.
	\item Service Composition.
\end{itemize}
\end{frame}
%------------------------------------------------

\begin{frame}
\frametitle{Supply Chain Information Transmission based on IoT and RFID}
\begin{block}{Information transmission plays very important role in-}
\begin{itemize}
	\item Demand forecast.
	\item Price Fluctuations.
	\item Limited supply and short term gaming.
\end{itemize}
\end{block}
\end{frame}
%------------------------------------------------

\begin{frame}
\frametitle{Supply Chain Information Transmission based on IoT and RFID}
\begin{block}{RFID system structure and its working}
\begin{figure}
\centering
\includegraphics[width=11cm,height=6cm]{RFID}
\caption{RFID System\cite{yan2009supply}}
\end{figure}
\end{block}
\end{frame}
%------------------------------------------------


\begin{frame}
\frametitle{Supply Chain Information Transmission based on IoT and RFID}
\begin{block}{EPC network structure and its working}
\begin{figure}
\centering
\includegraphics[width=11cm,height=6cm]{EPC}
\caption{EPC Network System\cite{yan2009supply}}
\end{figure}
\end{block}
\end{frame}
%------------------------------------------------

\begin{frame}
\frametitle{Supply Chain Information Transmission based on IoT and RFID}
\begin{block}{EPCglobal Network}
\begin{itemize}
	\item Current management organisation of IoT.
	\item Develops and manage EPC standards.
	\item Every enterprise must be a registered user of EPCglobal in order to apply IoT in their enterprise.\cite{wiki}
\end{itemize}
\end{block}
\end{frame}
%------------------------------------------------


\begin{frame}
\frametitle{Working of IoT using RFID in Supply Chain Information Transmission.}
\begin{block}{Procedure for start using IoT in a supply chain is as follows-}
\begin{itemize}
	\item Do registration at EPCglobal network.
	\item Now Supply chain members configure their internal internet of things equipments.
	\item For RFID systems, information coding rules in tags must adopt the standard developed by EPCglobal.
	\item RFID systems, EPC middlewares and EPC-IS are core technologies in this system.
	\item EPC middleware can be independently developed and compatibility between EPC middleware and background information system must be considered.
	\item With the services of local ONS (or Root ONS), supply chain members can get and update the product information in time.
\end{itemize}
\end{block}
\end{frame}
%------------------------------------------------

\begin{frame}
\frametitle{Application of IoT in Drug Supply Chain Information Transmission}
\begin{block}{Working of IoT in Drug Supply Chain}
\begin{figure}
\centering
\includegraphics[width=11cm,height=6cm]{drug}
\caption{IoT in drug supply chain\cite{yan2009supply}}
\end{figure}
\end{block}
\end{frame}
%------------------------------------------------



\begin{frame}
\frametitle{Benefits of IoT in Supply Chain Management}
\begin{block}{Benefits of IoT in Supply Chain Management are as follows-}
\begin{itemize}
	\item Internet of Things realize information visualization in a supply chain.
    \item Improves the level of information sharing among supply chain members.
    \item  Improves overall performance .
    \item Helps in decision making.
    \item IoT enhances  \say{Vendor relations} and \say{Customer retention}.
    \item Reduces machine downtime cost by planned and predictive maintenance. 
    \item Protects supply chain from bullwhip effect.
    \item Allows companies to live track their products.
\end{itemize}
\end{block}
\end{frame}
%------------------------------------------------

\begin{frame}
\frametitle{Benefits of IoT in Supply Chain Management}
\begin{block}{cont'd-}
\begin{itemize}
	\item Produces high quality, clean and reliable data.
    \item Visual inventory management allows the dynamic information about the inventory to be collected automatically over the IoT.
    \item Customers can understand the process of fabricating and transporting over the IoT.\cite{lou2011agile}
    \item Using IoT it is possible that a defect in a product can be serviced even before it come to notice of customer.    
\end{itemize}
\end{block}
\end{frame}
%------------------------------------------------

\begin{frame}
\frametitle{Conclusions}
\begin{enumerate}
	\item Internet of Things in Supply Chain allows fast, reliable and smoother information transmission.
	\item IoT enhances transparency in supply chain.
	\item IoT increases operational efficiencies and revenue opportunities for an organisation.
	\item Data generated from IoT is clean and precise, which can be further used for predictive analysis for inventory forecasting, demand forecasting etc.\cite{lancioni2000role}
	\item IoT offers \say{Perpetual connectivity}.
	\item IoT generates and process data real timely.
	\item Data collected can also be used for building new models.\cite{yan2009supply}
\end{enumerate}
\end{frame}
%------------------------------------------------



\begin{frame}[t, allowframebreaks]{References}
\bibliographystyle{amsalpha}	
\bibliography{ref}
	

\end{frame}


%------------------------------------------------
\begin{frame}
\frametitle{Extra}
\begin{block}{Bullwhip Effect}
\begin{figure}
\centering
\includegraphics[width=11cm,height=6cm]{bullwhip}
\caption{Bullwhip Effect}{Source: Internet}
\end{figure}
\end{block}
\end{frame}
%------------------------------------------------







\end{document} 